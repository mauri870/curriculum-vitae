% !TeX program = xelatex
% I made some modifications based on this template: 
%	https://github.com/Roald87/xelatex-cv-roald
% Thanks Roald87 for the awesome work!
\documentclass[]{cv-mauri}

\pagestyle{empty} % to remove the page numbers

\begin{document}

% This is the header of the first page, which contains your name and contact details. 
% \sep inserts a | between items. 
% You can use FontAwesome icons and use \FAspace after a font awesome icon to insert a predefined horizontal space after a font awesome icon.
\header{Mauri de Souza Nunes}{}{
	\faMapMarker \hspace{\FAspace} Brazil
}{
	\faMobile 
	\hspace{\FAspace} +55 54 981189792 \sep 
	\faEnvelope \hspace{\FAspace} \href{mailto:mauri870@gmail.com}{mauri870@gmail.com} \sep 
	\faLinkedinSquare \hspace{\FAspace} \href{http://linkedin.com/in/mauri870}{linkedin.com/in/mauri870} \sep
	\faGithub \hspace{\FAspace} \href{http://github.com/mauri870}{github.com/mauri870}
}

\textit{Work hard. Dream big.}

\section*{education}
% Use tabularcv environment to make a two column environment, left one for dates, right one for details of your education for example. 
% You can use the command \worktitle{Study name/Job title}{Location}.
% You can use the environment tabitemize to make a bulletpoint list inside the tabularcv environment.
\begin{tabularcv}
    2016-2019   &   \worktitle{Software Analysis and Development (AS)}{FTEC(BR)}
                    \\[\vspacepar] % Start new row with this
    2013-2014   &   \worktitle{Computing Technologist Incomplete}{FTEC(BR)}
\end{tabularcv}

\section*{work}
\begin{tabularcv}
    2016-2019   &   \worktitle{Software engineer}{FluxoTI / 3CPLUS Call Center Solutions}
                    \textbf{\textit{Remote / Full-time}}
                    \newline
                    Working with distributed systems, migrated the entire stack (it's a call center company) to kubernetes in GKE, refactored the arquitecture to be scalable horizontally, made some microservices using Go, NodeJS, Answering Machine Detection using convolutional neural networks with Tensorflow and all sorts of apps and sidecars, project management with SCRUM/XP, continuous integration with gitlab and k8s cluster management / devops stuff.
                    \textbf{Techs:} Go, NodeJS, gRPC, Protobuf, Python, Docker, Laravel, VueJS, MongoDB, PostgreSQL, Redis, AI, Tensorflow, Gitlab, Rancher, Kubernetes.
                    \\[\vspacepar]
                    \\[\vspacepar]
    2015-2016   &   \worktitle{Developer}{Digital Serra}
				    \textbf{\textit{Onsite / Full-time}}
                    \newline 
                    \textbf{\textit{Full-time}}
                    \newline
                    Web applications using PHP/Laravel and VueJS.
                    \\[\vspacepar]
                    \\[\vspacepar]
	2014-2015   &   \worktitle{IT Assistant}{Brazilian Army}
					\textbf{\textit{Onsite / Full-time}}
\end{tabularcv}

\section*{software skills}
    As a self-taught person, I learn a lot in my spare time, let's say that I never stop learning new things... I try to use most of what I learn at my job whenever possible, like AI, Go, Docker, Kubernetes and microservices. You can find some of my personal projects at \href{https://github.com/mauri870}{\color{maincolor}{https://github.com/mauri870}}, \href{https://mauri870.github.io}{\color{maincolor}{https://mauri870.github.io}} or at my blog \href{https://mauri870.github.io/blog}{\color{maincolor}{https://mauri870.github.io/blog}}.

\section*{operating systems}
	Just love Linux, I use Linux based operating systems since 2014 in a daily basis, currently with \href{https://github.com/mauri870/dot-files}{\color{maincolor}{Arch linux and i3wm }}. Besides personal use, I currently work with linux servers and on top of that Kubernetes and docker containers. Also, I have a thing for old operating systems, in particular the distributed arquitecture of Plan9 from Bell Labs and the awesome acme editor.

\section*{languages}
\begin{tabularcv}
    Portuguese\,(native) \\
    English\,(intermediate)
\end{tabularcv}

\section*{interests}
	\begin{description}
		\item [Data science, Fields of Deep Learning, AI] Signal Processing, Computer vision, NLP, Neural Networks, Tensorflow.
		\item [Languages] Go, Rust, NodeJS/JavaScript, Shell Script, Python, PHP.
		\item [Networking and Web / Apps] Microservices, GRPC, Distributed Systems, Electron, Cross-Platform Hybrid apps
		\item [DevOps] Kubernetes, Docker, Rancher, GCloud.
		\item [Low Level] ARM / Linux x64 assembly.
	\end{description}
\end{document}