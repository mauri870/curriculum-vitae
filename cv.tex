% !TeX program = xelatex
% I made some modifications based on this template: 
%	https://github.com/Roald87/xelatex-cv-roald
% Thanks Roald87 for the awesome work!
\documentclass[]{cv-mauri}

\pagestyle{empty} % to remove the page numbers

\begin{document}

% This is the header of the first page, which contains your name and contact details. 
% \sep inserts a | between items. 
% You can use FontAwesome icons and use \FAspace after a font awesome icon to insert a predefined horizontal space after a font awesome icon.
\header{Mauri de Souza Meneguzzo}{}{
	\faMapMarker \hspace{\FAspace} Brazil
}{
	\faMobile \hspace{\FAspace} +55 54 981189792 
	\sep 
	\faEnvelope \hspace{\FAspace} \href{mailto:mauri870@gmail.com}{mauri870@gmail.com} 
	\sep 
	\faLinkedinSquare \hspace{\FAspace} \href{http://linkedin.com/in/mauri870}{linkedin.com/in/mauri870} 
	\sep
	\faGithub \hspace{\FAspace} \href{http://github.com/mauri870}{github.com/mauri870}
	\sep
	\faGlobe \hspace{\FAspace} \href{https://mauri870.github.io}{mauri870.github.io}
}

\textit{Work hard. Dream big.}

\section*{work}
\begin{tabularcv}
	2021   &   \worktitle{Associate / Full Stack Software Engineer}{Bucks App / Startup}
					\textbf{\textit{Remote / Full-time / USA}}

			\begin{itemize}
				\item Mercury Raise Round IV for startups
				\item Developed backend systems using multiple AWS services (S3, Cognito, SES, SQS, Lambda, SAM).
				\item Developed a React Native mobile app for both Android and IOS.
				\item Integrating with financial solutions such as Dwolla, Checkbook and Plaid.
				\item  Developed an embedded Plaid Link for React Native that does not depend on native code.
			\end{itemize}
					\\[\vspacepar]
	
    2019-2021   &   \worktitle{CTO / Lead Software Engineer}{3CPLUS Call Center Solutions}
                    \textbf{\textit{Remote / Full-time}}
            \begin{itemize}
            	\item Managing a team of 4 engineers, including code review, interviewing potential team members and ensuring the team delivers sprints and OKR's on time.
            	\item Company reached a MRR growth of 120\% in 2020
            	\item More than a thousand concurrent users and thousands of simultaneous calls
            	\item Developed a high throughput bulk ingestion system in Go to feed realtime data into a PowerBI data lake for later mining and visualization.
            	\item Developed an Open Source Kubernetes operator to report services health to an external status page.
            	\item Rewrited the mailbox detection AI using Tensorflow 2, Tensorflow Serving and various improvements, resulted in a 10x speedup and 20x lower cpu usage.
            	\item Created an open source program(iowatch) in Rust to execute commands when files change to automate multiple tasks.
            	\item Added buffer reuse support to Google Storage Go to reduce allocations and GC pressure.
            	\item Developed mauri870/gcsfs, a bridge between golang's 1.16 new io/fs and Google Cloud Storage.
            \end{itemize}
                    \\[\vspacepar]

    2016-2019   &   \worktitle{Full Stack Software engineer}{3CPLUS Call Center Solutions}
		    		\textbf{\textit{Remote / Full-time}}
		    \begin{itemize}
		    	\item Leading the migration of a full stack project to Google Cloud Platform and Kubernetes.
		    	\item Used various GCP resources (GKE, Storage, IAM, MemoryStore, Compute Engine)
		    	\item Spliting a larger and complex system into a distributed architecture composed of smaller microservices on top of docker containers, scalable horizontally.
		    	\item Creating REST, GRPC and Websocket APIs in Go, NodeJS and PHP.
		    	\item Develop web applications and APIs with PHP and Laravel Framework, Redis, SQL and NoSQL, processing thousands of events per second.
		    	\item Multiple contributions to Laravel Framework, Laravel Horizon and the mongodb library.
		    	\item Codebase migration and adoption of PHP 7 and later 8.
		    	\item Developed an Asterisk PABX module in C and Rust to persist call recordings to Google Storage.
		    	\item Writing software and API documentation in Markdown, OpenAPI and PlantUML.
		    	\item Automating CI/CD with Gitlab and Docker, reduced build times to seconds and the deployment integrated with Kubernetes, mostly zero downtime.
		    	\item Designed and built a custom webphone with VoIP click-to-call and CRM integrations that runs as a web extension, PWA, hybrid app and webapp reusing the same codebase.
		    	\item Architected and applied Artificial Intelligence on the detection of mailboxes in realtime calls with Tensorflow and Python, greatly improving the performance and assertivity of our predictive dialing algorithm by more than 70\%.
		    	\item Developed a tool in Rust to filter and clear blacklisted data from CSV/TSV mailing files, increasing system assertivity.
		    	\item Created an event broker to serve realtime application events through Websocket and Server-Sent Events(SSE).
			\end{itemize}

		    \\[\vspacepar]
    2015-2016   &   \worktitle{Software Developer}{Freelancer}
            \begin{itemize}
                  	\item Web applications using PHP/Laravel and VueJS.
            \end{itemize}
                    
                    \\[\vspacepar]
	2014-2015   &   \worktitle{IT Assistant, Private Rank}{Brazilian Army}
					\textbf{\textit{Onsite / Full-time}}

			\begin{itemize}
				\item Linux and Windows troubleshooting and computer maintenance in general.
				\item Went to army reserve with a honorable mention.
			\end{itemize}
\end{tabularcv}

\section*{skills}
\begin{tabularcv}
	\textbf{Languages:} & TypeScript, Golang, JavaScript, SQL, Python, Bash \\
	\textbf{Devops:} & Docker, AWS, GCP, Gitlab \\
	\textbf{Processes:} & Agile, Scrum, Kanban, OKR's \\
\end{tabularcv}

\section*{education}
% Use tabularcv environment to make a two column environment, left one for dates, right one for details of your education for example. 
% You can use the command \worktitle{Study name/Job title}{Location}.
% You can use the environment tabitemize to make a bulletpoint list inside the tabularcv environment.
\begin{tabularcv}
    2017-2021   &   \worktitle{Software Analysis and Development}{FTEC}
                    \\[\vspacepar] % Start new row with this
    2013-2014   &   \worktitle{Computing Technologist (Incomplete)}{}
\end{tabularcv}

\section*{languages}
\begin{tabularcv}
	Portuguese\,(native) & \\
	English\,(fluent) &
\end{tabularcv}

%\end{description}
\end{document}
