% !TeX program = xelatex
% I made some modifications based on this template:
%	https://github.com/Roald87/xelatex-cv-roald
% Thanks Roald87 for the awesome work!
\documentclass[]{cv-mauri}

\pagestyle{empty} % to remove the page numbers

\begin{document}

% This is the header of the first page, which contains your name and contact details.
% \sep inserts a | between items.
% You can use FontAwesome icons and use \FAspace after a font awesome icon to insert a predefined horizontal space after a font awesome icon.
\header{Mauri de Souza Meneguzzo}{}{
	\hspace{\HSpacing} \Large Senior Software Engineer
	% \faMapMarker \hspace{\HSpacing} Remote, Brazil
}{
	% \faMobile \hspace{\FAspace} +01 02 0303030303
	% \sep
	\faEnvelope \hspace{\HSpacing} \href{mailto:mauri870@gmail.com}{mauri870@gmail.com}
	\sep
	\faLinkedin \hspace{\HSpacing} \href{http://linkedin.com/in/mauri870}{linkedin.com/in/mauri870}
	\sep
	\faGithub \hspace{\HSpacing} \href{http://github.com/mauri870}{github.com/mauri870}
	\sep
	\faGlobeAmericas \hspace{\HSpacing} \href{https://mauri870.com}{mauri870.com}
}

\section*{work}
\begin{tabularcv}
	2024 - ongoing   &   \worktitle{Senior Software Engineer, OpenTelemetry}{Elastic}
				\textbf{\textit{Remote / Full-time}}

		\begin{itemize}
			\item Part of the Ingest Data Plane team, enhanced observability in Elastic products.
			\item Developed Elastic Agent, Integrations, and Beats enhancements.
			\item Strong focus on open source and collaboration; most of my work is publicly available.
			\item Drove the usage of OpenTelemetry APIs and improvements to EDOT (Elastic's OpenTelemetry Collector).
			\item Contributed to the integration of an OpenTelemetry Collector within Elastic Agent, with the end goal of making it the default operation mode.
			\item Participated in the migration of standalone Beats to be OpenTelemetry Receivers.
			\item Elastic Agent in OTel mode achieved a consistent 50\% memory reduction and significant CPU savings across use cases by running Elastic Agent in OpenTelemetry mode, outperforming the previous implementation.
			\item Contributed to the OpenTelemetry project, focusing on the elasticsearchexporter and the contrib repository.
			\item Hands-on experience with the ELK Stack (Elasticsearch, Logstash, Kibana), gained through daily work and specialized training.
			\item Collaborated closely with cross-functional teams to ensure seamless OpenTelemetry integration across Elastic's ecosystem.
			\item Participated in the Support Dev Rotation, triaging complex bugs, addressing security vulnerabilities, and supporting escalated customer issues.
		\end{itemize}
				\\[\vspacepar]
	2022 - 2024   &   \worktitle{Full Stack Software Engineer}{Meisterwerk}
					\textbf{\textit{Remote / Full-time / Germany}}

			\begin{itemize}
				\item Collaborated with cross-functional teams to deliver high-quality software solutions that meet business and user requirements.
				\item Provided guidance and mentorship to the engineering team on best practices and standards.
				\item Designed and implemented HTTP and gRPC microservices using Golang, ensuring security, scalability, and maintainability of the system.
				\item Debugging and profiling techniques using tools such as Delve and pprof, finding memory and cpu bottlenecks.
				\item Improved app build times by 64\% by leveraging caching and parallelization.
				\item Exploratory Data Analysis and visualization using Python and Pandas.
				\item Implemented automated vulnerability scanning and static code analysis in CI/CD pipelines.
				\item Maintained a mobile app using React Native.
			\end{itemize}
					\\[\vspacepar]
	% 2022   &   \worktitle{Full Stack Software Engineer}{Watchcrunch}
	% 				\textbf{\textit{Remote / Full-time / USA}}

	% 		\begin{itemize}
	% 			\item Software development using the Laravel Framework
	% 		\end{itemize}
	% 				\\[\vspacepar]
	2021-2022   &   \worktitle{Associate / Full Stack Software Engineer}{Bucks App / Startup (finance)}
					\textbf{\textit{Remote / Full-time / USA}}

			\begin{itemize}
				\item Entered Mercury Raise Round IV for startups.
				\item Developed secure Go and NodeJS systems on AWS, integrated with financial solutions.
				\item Implemented cryptographycally secure systems to ensure data integrity and confidentiality.
				\item Mobile app development with React Native
			\end{itemize}
					\\[\vspacepar]

    2016-2021   &   \worktitle{CTO / Lead Software Engineer}{3CPLUS - SAAS (communications)}
                    \textbf{\textit{Remote / Full-time}}
            \begin{itemize}
            	\item Managing a team of 6 engineers, company achieved a MRR growth of 120\% in 2020.
            	\item Scaled a realtime system to handle thousands of concurrent users and millions of events per day.
            	\item Modeled a reliable ingestion system for a PowerBI data lake.
            	\item Independently built a open-source Kubernetes Health Check operator for automated status dashboards.
            	\item Improved AI projects with Tensorflow 2 and state-of-the-art techniques, resulting in a 10x speedup in inference and a 20x reduction in resource usage.
		    	\item Led a team of engineers on the project migration to GCP and Kubernetes.
		    	\item Adapted and extended all parts of the system to work in a distributed architecture, scalable horizontally.
		    	\item Developed REST, GRPC and Websocket APIs in Go, NodeJS and PHP.
		    	\item Deloped Asterisk PABX modules in Rust/C to optimize telephony routines.
		    	\item Software and API documentation with OpenAPI and PlantUML.
		    	\item Continuous Integration and Delivery with Gitlab, automated builds and deploys with zero downtime.
		    	\item Research on Deep Learning for mailbox detection resulted in a 70\% increase in accuracy.
		    	\item Developed an event broker to serve high-throughput events through different real-time protocols.
			\end{itemize}

		    \\[\vspacepar]
    2015-2016   &   \worktitle{Software Developer}{Freelancer}
            \begin{itemize}
                  	\item Web applications using PHP/Laravel and VueJS.
            \end{itemize}

                    \\[\vspacepar]
	2014-2015   &   \worktitle{IT Assistant, Private Rank}{6th Comms Bn of the 6th Div, Brazilian Army}
					\textbf{\textit{Onsite / Full-time}}

			\begin{itemize}
				\item Linux server administration, network management, and IT support.
				\item Collaborated on the development of a real-time dashboard system for the command center.
				\item Signal Corps, went to army reserve with a honorable mention.
			\end{itemize}
\end{tabularcv}

\section*{Projects}
\begin{tabularcv}
	{\Huge \color[HTML]{76E1FE} \FAB \char"E40F} &

	As a member of the Go project, I've made substantial contributions to the language, runtime and ecosystem.
	I was awarded the 2024 Google Open Source Peer Bonus for my work on the Go project.
	\\
	\includegraphics[scale=1.5]{images/k8s.png} &

	As a member of the Kubernetes project, I actively engage with the community to advance its
	development by fixing bugs, improving documentation, and participating in discussions to support
	feature development.

	\\
	\includegraphics[scale=0.35]{images/opentelemetry.png} &

	As a member of the OpenTelemetry project, I have helped to improve the quality and
	performance of the project by fixing bugs and adding new features. My main focus has been
	on the elasticsearchexporter and the contrib repository. I was nominated Top Contributor at KubeCon EU 2025.

	\\
	\includegraphics[scale=0.20]{images/elastic.png} &

	At Elastic, I was part of a small “skunk works” team focused on experimental projects, including
	EDOT (Elastic Agent as an OpenTelemetry Collector), the migration of Beats to OpenTelemetry receivers,
	and contributions to the OpenTelemetry ecosystem to help guide and advance it as the industry standard
	for observability. This work helped shape the future of observability and ensure Elastic's solutions
	remain relevant moving forward.

\end{tabularcv}

\section*{education}
% Use tabularcv environment to make a two column environment, left one for dates, right one for details of your education for example.
% You can use the command \worktitle{Study name/Job title}{Location}.
% You can use the environment tabitemize to make a bulletpoint list inside the tabularcv environment.
\begin{tabularcv}
	2024-2025	&   \worktitle{Postgraduate Specialization in Cybersecurity and Defense}{Uninter, Brazil}
					\begin{itemize}
						\item Introduction to Software Simulation in Combat and Armored Warfare - \textit{UFRGS, Brazilian Army}
					\end{itemize}
					\\[\vspacepar]
	2023-2024   	&   \worktitle{Postgraduate Specialization in Data Science and AI}{Uninter, Brazil}
					\newline Related coursework:
					\begin{itemize}
						\item Fundamentals of Neuroscience Program - \textit{Harvard, EdX}
						\item Integrative Neuroscience - \textit{UFRGS}
						\item Neurology of Learning - \textit{IFSUL}
						\item Sentiment Analysis and introduction to Natural Language Processing - \textit{UFRGS}
					\end{itemize}
                    \\[\vspacepar] % Start new row with this
    2017-2021   &   \worktitle{BTech in IT with a focus on Systems Analysis and Design}{Uniftec, Brazil}
                    \\[\vspacepar] % Start new row with this
\end{tabularcv}

\section*{languages}
\begin{tabularcv}
	Portuguese (native) & \\
	English (fluent) &
\end{tabularcv}

\section*{skills}
\begin{tabularcv}
	\textbf{Techs:} & Golang, Rust, PHP, Python, TypeScript/JS, C, Bash \\
	\textbf{Fields:} & Distributed Systems, AI, Cloud Computing, Web, Product Management, Cybersecurity \\
	\textbf{DevOps:} & Docker, Kubernetes, AWS, GCP, Gitlab \\
	\textbf{Processes:} & JIRA, Monday, Agile, Scrum, Kanban, OKR's \\
\end{tabularcv}


\vspace{1cm}
\textit{I propose to consider the question: "Can machines think?"} — Alan Mathison Turing

\textit{Work hard. Dream big. Stay humble.}

%\end{description}
\end{document}
